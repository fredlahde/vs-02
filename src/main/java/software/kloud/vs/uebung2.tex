\documentclass[12pt,a4paper]{report}
\usepackage[left=1cm,right=1cm,top=1.5cm,bottom=1cm,includeheadfoot]{geometry}

\usepackage[utf8]{inputenc}
\usepackage[ngerman]{babel}
\usepackage{booktabs}


\title{Verteilte Systeme}
\author{Lukas f. Paluch}
\date{}

\begin{document}

	\section{Modulus Ansatz}

		\begin{tabular}{|c|c|}
			intro.pdf & 1\\
			index.html & 2\\
			d41d8cd98f00b204e9800998ecf8427e & 0\\
			0cc175b9c0f1b6a831c399e269772661 & 1\\
			900150983cd24fb0d6963f7d28e17f72 & 1\\
			f96b697d7cb7938d525a2f31aaf161d0 & 2\\
			c3fcd3d76192e4007dfb496cca67e13b & 0\\
			c0008dfc-b5c6-4ac8-9b96-c1780084109b & 2\\
			fde20f54-6d1f-45a8-baf0-82d20b6a0c62 & 1\\
			10af8b96-bf4a-4567-9cf4-56646cfefb23 & 0\\
			cb4c2299-be2b-4673-ae76-0e98d1539833 & 1\\
			c01c429b-5a55-4930-be48-8b0e2ae4db5d & 0\\
		\end{tabular}

		\subsection{Was passiert, wenn s3.example.com als Server wegfällt?}

			Die Last wird auf die noch verfügbaren Server s2.example.com und s1.example.com verteilt.

	\section{HRW-Ansatz}

		\subsection{Die Verteilung auf 3 Nodes}

		\begin{tabular}{|c|r|r|}
			Node & Anzahl & Prozent \\ \midrule
			n0 & 33\% & 3372 \\ \hline
			n1 & 33\% & 3325 \\ \hline
			n2 & 33\% & 3303 \\ \hline
		\end{tabular}


		\subsection{Die Verteilung auf 2 Nodes}

		\begin{tabular}{|c|r|r|}
			Node & Anzahl & Prozent \\ \midrule
			n0 & 50\% & 4996 \\ \hline
			n1 & 50\% & 5004 \\ \hline
		\end{tabular}
		Wenn s3 aus der Liste enfernt wird verteilt sich die Last gleichmäßig auf die verbliebenen 2 Nodes.

		\subsection{Die Verteilung auf 4 Nodes}

		\begin{tabular}{|c|r|r|}
			Node & Anzahl & Prozent \\ \midrule
			n0 & 25\% & 2469 \\ \hline
			n1 & 25\% & 2523 \\ \hline
			n2 & 25\% & 2545 \\ \hline
			n3 & 25\% & 2463 \\ \hline
		\end{tabular}
		Wenn s4 zur Liste hinzugefügt wird verteilt sich die Last gleichmäßig auf die 4 Nodes.

\end{document}

